\section{Preparación de datos}

\subsection{Selección de variables}
Se eliminaron variables con baja contribución y/o redundancia, manteniendo la variable objetivo al final del flujo para facilitar el procesamiento.

\subsection{Descripción estadística}
Se realizaron inspecciones de tipos, conteos, distribución de clases y porcentaje de nulos por variable.

\subsection{Limpieza de atípicos}
No se aplicó una eliminación masiva de atípicos por tratarse mayoritariamente de variables categóricas codificadas y binarias. En variables numéricas se verificó consistencia de rangos.

\subsection{Limpieza de nulos}
Se eliminaron registros con alta proporción de nulos y se imputaron nulos categóricos usando KNN sobre codificación temporal.

\subsection{Creación de nuevas variables}
No se crearon variables sintéticas adicionales; se aplicó codificación one-hot para variables categóricas.

\subsection{Análisis de relaciones}
Se revisó comportamiento de la variable objetivo antes y después del balanceo, y relación de síntomas/atributos con la clase.

\subsection{Reducción de dimensiones}
En la versión actual no se aplicó PCA. La decisión se sustentó en mantener interpretabilidad y en dimensionalidad manejable tras selección de variables.

\subsection{Balanceo}
Se aplicó SMOTE para balancear las clases de hospitalización y reducir sesgo por desbalance de clase.