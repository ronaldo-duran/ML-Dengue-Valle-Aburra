\section{Despliegue}

\subsection{Predicción de datos futuros}
Se evaluó el modelo final sobre \textbf{50 registros reales reservados} y no usados en entrenamiento ni en el tuning.

\subsection{Construcción del conjunto de entrada}
El conjunto de entrada se construyó desde \texttt{df\_future\_raw}, conservando su etiqueta real para contrastar predicción vs realidad.

\subsection{Preparación del conjunto nuevo}
Se reindexaron columnas según \texttt{model\_input\_columns.joblib} para garantizar compatibilidad con el pipeline final.

\subsection{Predicción e interpretación}
Las salidas del modelo se interpretaron como:
\begin{itemize}
    \item 1: Hospitalización Sí
    \item 0: Hospitalización No
\end{itemize}

En estos 50 registros se obtuvo:
\begin{itemize}
    \item Aciertos: 39
    \item Fallos: 11
    \item Accuracy: 0.78
    \item F1-macro: 0.7329
    \item Recall clase positiva: 0.5625
    \item Precision clase positiva: 0.6923
\end{itemize}

\subsection{Aplicación web con Streamlit}
Se implementó una app en \texttt{streamlit\_app/app.py} con formulario manual de variables clínicas/demográficas, que genera la predicción y la interpretación de hospitalización esperada.

\textbf{Pendiente para entrega final:} incluir captura de pantalla funcionando y URL pública de Streamlit Cloud.