\section{Modelamiento y evaluación}

\subsection{Configuración y selección de métodos}
Se trabajó con dos familias:
\begin{itemize}
    \item \textbf{Modelos clásicos (5):} Regresión Logística, Árbol de Decisión, KNN, SVM y MLP.
    \item \textbf{Ensembles (3):} VotingClassifier (votación), Random Forest (bagging) y XGBoost (boosting).
\end{itemize}

\subsection{Ajuste de hiperparámetros con 70\% de datos + CV}
Se realizó partición estratificada 70/30 sobre el conjunto balanceado final: \texttt{X\_train=(16207,46)} y \texttt{X\_test=(6947,46)}. El ajuste de hiperparámetros se ejecutó exclusivamente sobre entrenamiento mediante \texttt{GridSearchCV} con \texttt{StratifiedKFold(n\_splits=3)}.

\subsubsection{Justificación de la métrica}
Se utilizó \textbf{F1-macro} como métrica principal para balancear desempeño entre clases y no favorecer una clase particular.

\subsubsection{Ajuste de modelos clásicos}
Se ajustaron hiperparámetros de los 5 modelos clásicos definidos con búsqueda exhaustiva por grilla.

\subsubsection{Ajuste de modelos de ensamble}
Se ajustaron hiperparámetros de VotingClassifier, Random Forest y XGBoost con el mismo esquema de validación cruzada estratificada.

\subsection{Medida de calidad del modelo}
La evaluación final se hizo sobre el 30\% de prueba no usado en tuning. Para el mejor modelo (XGBoost) la matriz de confusión en test fue:
\[
\begin{bmatrix}
3025 & 449 \\
793 & 2680
\end{bmatrix}
\]

\begin{table}[H]
\centering
\begin{tabular}{llll}
\toprule
Modelo & Tipo & F1-macro (CV train) & F1-macro (test) \\
\midrule
Regresión Logística & Clásico & 0.7383 & 0.7488 \\
Árbol de Decisión & Clásico & 0.7670 & 0.7761 \\
KNN & Clásico & 0.7295 & 0.7394 \\
SVM & Clásico & 0.7850 & 0.7963 \\
Red Neuronal MLP & Clásico & 0.7737 & 0.7884 \\
VotingClassifier & Ensamble (Votación) & 0.7702 & 0.7855 \\
Random Forest & Ensamble (Bagging) & 0.7958 & 0.8077 \\
XGBoost & Ensamble (Boosting) & 0.8143 & 0.8208 \\
\bottomrule
\end{tabular}
\caption{Resultados comparativos de modelos (bloque formal 70/30)}
\end{table}

\noindent\textbf{Mejor modelo:} XGBoost, con \texttt{test\_f1\_macro=0.8208} y \texttt{test\_accuracy=0.8212}.

\subsubsection{Selección del mejor modelo}
El mejor modelo se seleccionó por el mayor valor de \texttt{test\_f1\_macro}. Además, se reportaron recall, precision y accuracy en test y se exportaron los artefactos \texttt{best\_model\_formal\_7030.joblib} y \texttt{model\_input\_columns.joblib}.